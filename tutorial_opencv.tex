\documentclass[11pt,a4paper,onecolumn]
{article}

\usepackage[utf8]{inputenc}
\usepackage{fontenc}
\usepackage[dvips]{graphicx}
\usepackage[spanish]{babel}
\usepackage{booktabs}
\usepackage{amsmath}
\usepackage{amssymb}
\usepackage{indentfirst}
\usepackage{hyperref}
\parindent=0in
\parskip=5pt

\addtolength{\voffset}{-2cm}
\addtolength{\textheight}{2cm}
%\addtolength{\hoffset}{-0.5cm}
\addtolength{\textwidth}{2cm}
%\usepackage{pstricks}

\newcommand{\tc}[1]{\texttt{#1}}

\newlength{\drop} % generate specific amount of whitespace
\drop=1em
\newcommand{\Ws}{\hspace{\drop}}

\title{
	\small FICH, UNL - Departamento de Informática - Ingeniería Informática\\
	\vspace{0.5cm}
	\Large \bf{Procesamiento Digital}\\
	\bf{de Imágenes}\\
	\vspace{0.5cm}
	\normalsize Introducción a la librería OpenCV y puesta en funcionamiento\\
	% \vspace{0.5cm}
	% \Large \textsf{Operaciones puntuales}
 }
\author{\small 2015}
\date{}

\begin{document}
	\maketitle

	\section{Introducción}
Este documento presenta una reseña de la librería OpenCV para procesamiento de imágenes en C++. Se explica brevemente su instalación, compilación y las funciones fundamentales para cargar una imagen de archivo, obtener información de la misma, visualizarla y otras cuestiones de manejo básico, 

La librería y su documentación está disponible en: \url{http://opencv.org/}. En nuestras clases usaremos la versión 2.x. % OO, RAII, estable

	\section{Instalación}
\url{http://docs.opencv.org/doc/tutorials/introduction/linux_install/linux_install.html}\\
\url{http://docs.opencv.org/doc/tutorials/introduction/windows_install/windows_install.html}

	\section{Compilación}
\tc{OpenCV} se divide en módulos según la funcionalidad que estos proveen, por ejemplo el módulo \tc{highgui} permite realizar interfaces gráficas, leer y escribir imágenes a disco y capturar video desde una cámara.
Para evitar la búsqueda de los módulos correspondientes a las funciones a utilizar, se incluirán todos ellos mediante\\
\verb|  #include<opencv2/opencv.hpp>| %\verb|#include<opencv/cv.hpp>|

Utilizando el programa \tc{pkg-config} se resolverán las dependencias de \tc{OpenCV} y se enlazará contra \emph{todos} los módulos (no existe penalidad)\\
\verb|  g++ prog.cpp $(pkg-config --libs opencv) -o prog|


	\section{Manejo básico de una imagen}
\begin{itemize}
		\item Crear una imagen vacía: \\
			\verb|  cv::Mat img(filas, columnas, tipo, valor);|\\
			\verb|  cv::Mat img = cv::Mat::zeros(filas, columnas, tipo);|

			\begin{itemize}
				\item \tc{valor}: Inicialización de los píxeles, es de tipo \tc{cv::Scalar}.
				\item \tc{tipo}: Define la forma de representación de la matriz. Se compone de la siguiente manera\\
					\verb|  CV_[n bits por elemento][signo][prefijo de tipo]C([n de canales])|\\
					Así, \tc{CV\_8UC(3)} indica que se almacenará la imagen utilizando \tc{unsigned char} de 8 bits, y que cada píxel se define con 3 canales.
			\end{itemize}


			Ejemplos:
			\begin{itemize}
				\item \verb|cv::Mat img(640, 480, CV_8UC(1))|: imagen en tonos de grises, inicialmente negra.
				\item \verb|cv::Mat img(640, 480, CV_8UC(3), cv::Scalar(0,0,255))|: imagen color, inicialmente roja (BGR).
				\item \verb|Mat C = (Mat_<double>(3,3) << 0, -1, 0, -1, 5, -1, 0, -1, 0)|: imagen de $3\times3$ utilizada como máscara para filtros
			\end{itemize}

	\item Crear una imagen cargándola de un archivo: \\
		\verb|  cv::Mat img = cv::imread("archivo.ext");|\\
		\verb|  cv::Mat img = cv::imread("archivo.ext", CV_LOAD_IMAGE_GRAYSCALE);| (convierte a escala de grises)

	\item Campos de la estructura \tc{Mat}: accesibles a través de las siguientes funciones, mediante la instrucción \tc{img.funcion}:
		\begin{itemize}
			\item \tc{columns}: ancho, número de columnas.
			\item \tc{rows}: alto, número de filas.
			\item \tc{channels}: dimensión del pixel, número de canales.
			\item \tc{depth()}: especificación de tipo.
			\item \tc{at<tipo>(R,C)}: acceso a un píxel, el tipo debe corresponderse con el la matriz.
			\item \tc{ptr<tipo>(R)}: puntero a una fila, el tipo debe corresponderse con el la matriz.
		\end{itemize}
		La matriz se accede desde el elemento (0,0) en la esquina superior izquierda, hasta el elemento (ancho-1,alto-1) en la esquina inferior derecha.

	\item Visualización de una imagen:\\
		Para mostrar las imágenes es necesario crear ventanas, estas son accedidas luego mediante el nombre:\\
		\verb|  cv::namedWindow("nombre");|\\
		\verb|  cv::imshow("nombre", img);|
		%A una ventana se pueden asociar barras de desplazamiento mediante la función \tc{createTrackbar()}

	\item Grabación de una imagen:\\
		Para guardar en disco una imagen, simplemente se llama a la función:\\
		\verb|  cv::imwrite("nombre.ext", img);|

		Las imágenes a color se consideran en orden BGR.

		Se puede utilizar un parámetro adicional para especificar opciones específicas del formato de imagen, como ser la compresión.

	\item Video:\\
		La clase \tc{cv::VideoCapture} proporciona una interfaz para capturar video desde una cámara o desde archivo.\\
		\verb|  cv::VideoCapture video(0);| (cámara por defecto)\\
		\verb|  cv::VideoCapture video("archivo.ext");| (lee desde un archivo de video)\\
		\verb|  cv::VideoCapture video("img%02d.jpg");| (lee una secuencia de imágenes: \tc{img00.jpg}, \tc{img01.jpg}, \ldots)

		Para obtener el siguiente fotograma, simplemente se utiliza\\
		\verb|  video >> frame;|\\
		donde \tc{frame} es de tipo \tc{cv::Mat}

		La cantidad de canales de la imagen dependerá del dispositivo de captura, pudiendo luego convertirse mediante la función \tc{cv::cvtColor()}
\end{itemize}

		\subsection{Copia}
			Las imágenes son representadas mediante una cabecera, que contiene información como el tamaño y tipo de la imagen, y un puntero inteligente que apunta a la zona de memoria donde están almacenados los píxeles.

			Al utilizar el operador de asignación o el constructor de copia, se copian la cabecera y el valor del puntero. Es decir, se realiza una \emph{shallow copy}. Este comportamiento se asemeja al que se observa en los objetos de \emph{java} o \emph{smalltalk}.

			Como consecuencia, si se pasa una \tc{cv::Mat} como parámetro a una función, es indistinto si el pasaje es por copia o referencia, e incluso la especificación \tc{const} deja de tener valor.
			Cualquier cambio en el parámetro de la función se verá reflejado en la variable usada en la llamada.

			Si se requiere trabajar con una copia verdadera (una \emph{deep copy}), se deberán utilizar los comandos \tc{clone()} o \tc{copyTo()}\\
			\verb|  b = a.clone();|\\
			\verb|  c.copyTo(d);|

	%\section{Funciones adicionales}
	\section{Ejemplo completo}
	%TODO: insertar canny
\begin{verbatim}
#include <opencv2/opencv.hpp>

int main(int argc, char **argv){
  cv::VideoCapture capture(0); //cámara por defecto
  if(not capture.isOpened()) return 1;
  cv::Mat frame, edge;

  do{
      capture>>frame;  //siguiente fotograma
      cv::cvtColor(frame, edge, CV_BGR2GRAY); //conversión a escala de grises
      cv::GaussianBlur(edge, edge, cv::Size(11,11), 2.5, 0.5);
      cv::Canny(edge, edge, 0, 30, 3);
      cv::imshow("original", frame);
      cv::imshow("borde", edge);
  }while( cv::waitKey(30)==-1 ); //salir cuando se presione una tecla

  return 0;
}


\end{verbatim}
	\section{Diferencias con CImg}
		Algunas diferencias que podrían afectar el cursado
		\begin{itemize}
			\item \tc{OpenCV} no soporta el formato \tc{gif}. Puede utilizarse el programa \tc{imagemagick} o similar para convertir a un formato manejable (como ser \tc{tiff}).\\
				\verb|  convert archivo.gif archivo.tiff|
			\item \tc{OpenCV} maneja las representaciones de color \tc{HSV} y \tc{HLS}, pero no \tc{HSI}.
			\item Una ventana puede mostrar solo una imagen a la vez.
		\end{itemize}
	\section{Comentarios finales}
		Este documento es de libre distribución y reproducción total o parcial por cualquier medio. Comentarios y sugerencias a los contactos de pdi-fich.wikidot.com.
\end{document}


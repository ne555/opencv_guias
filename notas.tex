\documentclass[oneside,a4paper]{book}
\usepackage[utf8]{inputenc}
\usepackage{fontenc}
\usepackage[spanish]{babel}
\usepackage{hyperref}
\usepackage{listings}
\lstset{
	language=C++,
	texcl=true
}
%\usepackage[a5paper]{geometry}

\title{OpenCV: notas}
\author{}
\date{18 de febrero de 2016}

\usepackage{listings}
\newcommand{\tc}[1]{\texttt{#1}}

\begin{document}
	\maketitle
	\tableofcontents
	%\setcounter{chapter}{-1}
	\frontmatter
	\let\theoldsection\thesection
	\makeatletter
		\renewcommand\thesection{\@arabic \c@section}
	\makeatother
\chapter{Introducción}
Este documento presenta una reseña de la librería OpenCV para procesamiento de imágenes en C++. Se explica brevemente su instalación, compilación y las funciones fundamentales para cargar una imagen de archivo, obtener información de la misma, visualizarla y otras cuestiones de manejo básico,

La librería y su documentación está disponible en: \url{http://opencv.org/}. En nuestras clases usaremos la versión 2.x. % OO, RAII, estable

	\section{Instalación}
\noindent\url{http://docs.opencv.org/doc/tutorials/introduction/linux_install/linux_install.html}\\
\url{http://docs.opencv.org/doc/tutorials/introduction/windows_install/windows_install.html}

Existe un complemento para ZinjaI que contiene la librería en la versión 2.4.10 compilada con MinGW, autocompletado y plantilla de projecto
\url{http://zinjai.sourceforge.net/}

	\section{Compilación}
\tc{OpenCV} se divide en módulos según la funcionalidad que estos proveen, por ejemplo el módulo \tc{highgui} permite realizar interfaces gráficas, leer y escribir imágenes a disco y capturar video desde una cámara.
Para evitar la búsqueda de los módulos correspondientes a las funciones a utilizar, se incluirán todos ellos mediante\\
\verb|  #include<opencv2/opencv.hpp>| %\verb|#include<opencv/cv.hpp>|

Utilizando el programa \tc{pkg-config} se resolverán las dependencias de \tc{OpenCV} y se enlazará contra \emph{todos} los módulos (no existe penalidad)\\
\verb|  g++ prog.cpp $(pkg-config --libs opencv) -o prog|


	\section{Manejo básico de una imagen}
\begin{itemize}
		\item Crear una imagen vacía: \\
			\verb|  cv::Mat img(filas, columnas, tipo, valor);|\\
			\verb|  cv::Mat img = cv::Mat::zeros(filas, columnas, tipo);|

			\begin{itemize}
				\item \tc{valor}: Inicialización de los píxeles, es de tipo \tc{cv::Scalar}.
				\item \tc{tipo}: Define la forma de representación de la matriz. Se compone de la siguiente manera\\
					\verb|  CV_[n bits por elemento][signo][prefijo de tipo]C([n de canales])|\\
					Así, \tc{CV\_8UC(3)} indica que se almacenará la imagen utilizando \tc{unsigned char} de 8 bits, y que cada píxel se define con 3 canales.
			\end{itemize}


			Ejemplos:
			\begin{itemize}
				\item \verb|cv::Mat img(640, 480, CV_8UC(1))|: imagen en tonos de grises, inicialmente negra.
				\item \verb|cv::Mat img(640, 480, CV_8UC(3), cv::Scalar(0,0,255))|: imagen color, inicialmente roja (BGR).
				\item \verb|Mat C = (Mat_<double>(3,3) << 0, -1, 0, -1, 5, -1, 0, -1, 0)|: imagen de $3\times3$ utilizada como máscara para filtros
			\end{itemize}

	\item Crear una imagen cargándola de un archivo: \\
		\verb|  cv::Mat img = cv::imread("archivo.ext");|\\
		\verb|  cv::Mat img = cv::imread("archivo.ext", CV_LOAD_IMAGE_GRAYSCALE);| (convierte a escala de grises)

	\item Campos de la estructura \tc{Mat}: accesibles a través de las siguientes funciones, mediante la instrucción \tc{img.funcion}:
		\begin{itemize}
			\item \tc{columns}: ancho, número de columnas.
			\item \tc{rows}: alto, número de filas.
			\item \tc{channels}: dimensión del pixel, número de canales.
			\item \tc{depth()}: especificación de tipo.
			\item \tc{at<tipo>(R,C)}: acceso a un píxel, el tipo debe corresponderse con el la matriz.
			\item \tc{ptr<tipo>(R)}: puntero a una fila, el tipo debe corresponderse con el la matriz.
		\end{itemize}
		La matriz se accede desde el elemento (0,0) en la esquina superior izquierda, hasta el elemento (ancho-1,alto-1) en la esquina inferior derecha.

	\item Visualización de una imagen:\\
		Para mostrar las imágenes es necesario crear ventanas, estas son accedidas luego mediante el nombre:\\
		\verb|  cv::namedWindow("nombre");|\\
		\verb|  cv::imshow("nombre", img);|
		%A una ventana se pueden asociar barras de desplazamiento mediante la función \tc{createTrackbar()}

	\item Grabación de una imagen:\\
		Para guardar en disco una imagen, simplemente se llama a la función:\\
		\verb|  cv::imwrite("nombre.ext", img);|

		Las imágenes a color se consideran en orden BGR.

		Se puede utilizar un parámetro adicional para especificar opciones específicas del formato de imagen, como ser la compresión.

	\item Video:\\
		La clase \tc{cv::VideoCapture} proporciona una interfaz para capturar video desde una cámara o desde archivo.\\
		\verb|  cv::VideoCapture video(0);| (cámara por defecto)\\
		\verb|  cv::VideoCapture video("archivo.ext");| (lee desde un archivo de video)\\
		\verb|  cv::VideoCapture video("img%02d.jpg");| (lee una secuencia de imágenes: \tc{img00.jpg}, \tc{img01.jpg}, \ldots)

		Para obtener el siguiente fotograma, simplemente se utiliza\\
		\verb|  video >> frame;|\\
		donde \tc{frame} es de tipo \tc{cv::Mat}

		La cantidad de canales de la imagen dependerá del dispositivo de captura, pudiendo luego convertirse mediante la función \tc{cv::cvtColor()}
\end{itemize}

		\subsection{Copia}
			Las imágenes son representadas mediante una cabecera, que contiene información como el tamaño y tipo de la imagen, y un puntero inteligente que apunta a la zona de memoria donde están almacenados los píxeles.

			Al utilizar el operador de asignación o el constructor de copia, se copian la cabecera y el valor del puntero. Es decir, se realiza una \emph{shallow copy}. Este comportamiento se asemeja al que se observa en los objetos de \emph{java}, \emph{python} o \emph{smalltalk}.

			Como consecuencia, si se pasa una \tc{cv::Mat} como parámetro a una función, es prácticamente indistinto si el pasaje es por copia o referencia.
			Cualquier cambio en el parámetro de la función se verá reflejado en la variable usada en la llamada (excepto realocaciones).

			Si se requiere trabajar con una copia verdadera (una \emph{deep copy}), se deberán utilizar los comandos \tc{clone()} o \tc{copyTo()}\\
			\verb|  b = a.clone();|\\
			\verb|  c.copyTo(d);|

	%\section{Funciones adicionales}
	\section{Ejemplo completo}
\begin{lstlisting}
#include <opencv2/opencv.hpp>

int main(int argc, char **argv){
  cv::VideoCapture capture(0); //cámara por defecto
  if(not capture.isOpened()) //error
    return 1;
  cv::Mat frame, edge;

  do{
    capture>>frame;  //siguiente fotograma
    cv::cvtColor(frame, edge, CV_BGR2GRAY); //conversión a escala de grises
    cv::GaussianBlur(edge, edge, cv::Size(11,11), 2.5, 0.5); //reducción de ruido
    cv::Canny(edge, edge, 0, 30, 3); //detección de bordes
    cv::imshow("original", frame);
    cv::imshow("borde", edge);
  }while( cv::waitKey(30)==-1 ); //salir cuando se presione una tecla

  return 0;
}
\end{lstlisting}
	\section{Diferencias con CImg}
		Algunas diferencias que podrían afectar el cursado
		\begin{itemize}
			\item \tc{OpenCV} no soporta el formato \tc{gif}. Puede utilizarse el programa \tc{imagemagick} o similar para convertir a un formato manejable (como ser \tc{tiff}).\\
				\verb|  convert archivo.gif archivo.tiff|
			\item \tc{OpenCV} maneja las representaciones de color \tc{HSV} y \tc{HLS}, pero no \tc{HSI}.
			\item Una ventana puede mostrar solo una imagen a la vez.
		\end{itemize}
	\section{Comentarios finales}
		Este documento es de libre distribución y reproducción total o parcial por cualquier medio. Comentarios y sugerencias a los contactos de pdi-fich.wikidot.com.

	\mainmatter
	\let\thesection\theoldsection
\chapter{Operatoria básica}
	\begin{itemize}
		\item Carga: comando \verb|cv::imread()|
			Se especifica si se quiere escala de grises o a color (BGR),
			y la profundidad (8 bits por defecto)

		\item Formato: los píxeles se almacenan en posiciones contiguas de memoria,
			siguiendo el orden de las filas.
			Toda la información de un píxel se encuentra contigua,
			es decir los canales se encuentran intercalados (\emph{interleaved})
			B1G1R1B2G2R2\ldots

		\item Información estadística
			\verb|cv::meanStdDev(), cv::minMaxLoc(), cv::Size::area()|

		\item Visualización
			\verb|cv::namedWindow(), cv::imshow()|
			Se crea una ventana a la que luego se refiere por el nombre.
			Indica cómo actua una operación de \emph{resize}: no hacer nada, mantener aspecto.

			Sólo se permite una imagen por ventana, la imagen debe ser de escala de grises o color en BGR,
			y los valores de los píxeles deben encontrarse en el rango $[0,1]$ para flotantes (si son enteros toma todo el rango)

		\item Gráficos: existen funciones para dibujar líneas y poligonales
			%que la cátedra provea pdi::plot()
			%https://code.google.com/p/cvplot/

		\item Operaciones puntuales: puede tratarse a la matriz como un todo,
			siendo las operaciones aplicadas por igual a todos los píxeles (eficiente).
			O puede recorrerse mediante un bucle y acceder a los píxeles individualmente \verb|.at<T>()|

		\item Copia: debido a que internamente se usan punteros inteligentes, \emph{asignación no es copia}.
			Para copiar se debe utilizar el comando \verb|.clone()|, lo cual creará una nueva imagen.

			Si lo que se desea es igualar los valores de los píxeles, se usa el comando \verb|.copyTo()|,
			pero el destino debe tener suficiente capacidad, caso contrario creará una nueva imagen.
			Además, permite el uso de una máscara para establecer los píxeles a copiar.

		\item Región de interés (ROI):
			OpenCV permite construir una imagen a partir de una región en otra.
			Es una operación muy eficiente ya que no hay ninguna copia involucrada,
			sino simplemente la creación de una cabecera para interactuar con la región.
			\verb|cv::Mat::operator()|

			Esta región puede delimitarse mediante el rango de filas y columnas que abarca,
			o mediante un rectángulo.
			Debe tenerse especial cuidado en que el rectángulo es geométrico,
			por lo que trabaja en coordenadas (x,y) que no es equivalente a (fila, columna), sino a
			\[
				x = \mathrm{col}\qquad
				y = \mathrm{TotalRows}-\mathrm{row}
			\]
			\begin{lstlisting}
roi = image(
  cv::Range(f_inic, f_fin), //filas
  cv::Range(c_inic, c_fin) //columnas
);
roi = image(
  cv::Rect(x,y, ancho, alto)
);
			\end{lstlisting}
	\end{itemize}

	\section{Callbacks}
		Mediante el módulo HighGUI, OpenCV provee interfaces para la creación de \emph{trackbars},
		manejar eventos del ratón y comandos del teclado.

		\begin{itemize}
			\item Trackbars:
			Los trackbars se asocian a una ventana mediante \verb|cv::createTrackbar()|,
			se establece una variable a modificar, el rango en que actúa, y es posible asignarle un callback.

			La variable que afectan es de tipo \verb|int|,
			pero es posible usarla de paso intermedio y obtener un valor flotante
			mediante una función de escalado.

		\item Eventos del ratón \verb|cv::setMouseCallback()|:
			Puede consultarse el estado de los botones.

			Salvedad: una matriz tiene filas y columnas,
			el punto fila=0, columna=0 corresponde a la esquina superior izquierda.
			Cuando se trabaja con el mouse o con figuras geométricas (\verb|cv::Point|),
			se establece un sistema coordenado (x,y) donde el punto (0,0) corresponde a la esquina inferior izquierda. Así, se tiene la equivalencia
			\[
				x = \mathrm{col}\qquad
				y = \mathrm{TotalRows}-\mathrm{row}
			\]

		\item Eventos del teclado:
			No están asociados a ninguna ventana en particular.
			\verb|cv::waitKey()|

		\end{itemize}
\chapter{Operaciones puntuales}
	OpenCV provee de sobrecarga de operadores que corresponden a operaciones de matrices.
	%OpenCV provee de sobrecarga de operadores que traducen las operaciones efectuadas en la matriz
	%a operaciones equivalentes en cada píxel.

	Con esto hay que considerar
	\begin{itemize}
		\item \verb|A*B| donde tanto \texttt{A} como \texttt{B} son \texttt{cv::Mat}
			realiza una multiplicación de matrices %(sólo con flotantes o complejos). TODO
			Para multiplicar píxel a píxel se utiliza la función \verb|cv::multiply()|
			o \verb|.mul()|

		\item \verb|CV_64FC(2)| no refiere a una matriz de complejos, sino a una matriz de dos canales.
			Al aplicar \verb|cv::multiply()| se opera en cada canal por separado.
			Usar \verb|cv::mulSpectrum()| si se quiere operar como complejo (con posibilidad de usar el conjugado).
	\end{itemize}

	\section{Aplicación de una LUT}
		En caso de que la imagen sea de 8bits se puede utilizar una LUT para realizar operaciones.
		\verb|cv::LUT( image, lut, transform );|

	\section{Ruido}
		\verb|cv::randn(), cv::randu()|

	\section{Umbralizado}
		\verb|cv::threshold()|
		También pueden ocuparse operadores relacionales \verb|<,>|

		Una vez que se tiene la máscara, se puede extraer los píxeles de interés mediante \verb|.copyTo()|

\chapter{Filtrado espacial}
	Debido a que la librería es bastante genérica, se proveen \emph{wrappers} para el cálculo de histogramas y la convolución y funciones para graficar.

	Ecualización \verb|equalizeHist()|

	Comparación \verb|compareHist()|, como función de distancia puede utilizarse la correlación.

	Para obtener kernel gaussiano \verb|cv::getGaussianKernel(size, sigma, CV_32F)|. Esto genera un vector, entonces se aplica el producto externo para obtener la matriz.

	Convolución con \verb|filter2D()|.
	También se puede utilizar \verb|sepFilter2D()| trabajando con filtros 1D.

	Filtrado gaussiano \verb|GaussianBlur()|
	%puede conseguirse un sigma ``óptimo'' si se dejan sigmaX y sigmaY como 0.


\chapter{Color}
	OpenCV no soporta el modelo HSI sino el HSV.
	Para cambiar entre modelos

	\begin{lstlisting}
cvtColor(in, out, CV_BGR2HSV)
cvtColor(in, out, CV_HSV2BGR)
	\end{lstlisting}

	Las imágenes a color que se obtienen de archivos o dispositivos de capturan, o las que se usan en la visualización, se encuentran en el modelo BGR.
	En almacenamiento, se tiene la terna bgr correspondiente al píxel 0, y luego la terna bgr correspondiente al píxel 1.
	
	Para acceder a estos valores se puede utilizar la función \verb|at()| donde el parámetro de tipo de dato será \verb|Vec3b|

	Para trabajar con los canales por separado, se tiene la función \verb|split()|
	y luego usar \verb|merge()| para volverlos a unir.

	La conversión BGR2HSV tiene pérdida si se trabaja con 8bits, no así en 32 bits.

	En HSV 8bits, el $hue = [0; 180)$, para su correspondencia en grados se multiplica por $2$, sino abarca el rango $[0; 360)$

	Se provee la función \verb|pdi::load_colormap()| para cargar la paleta de colores, luego utilizar \verb|cv::LUT()| para aplicarla.

\chapter{Filtrado frecuencial}
	Proceso de filtrado:
	\begin{enumerate}
		\item convertir a 32F o 64F: \verb|image.convertTo(image, CV_32F, 1./255)|
		\item ajustar tamaño para que las operaciones sean más eficientes %funciona sin esto
			\verb|cv::getOptimalDFTSize()|\\
			\verb|cv::copyMakeBorder()|
			\begin{lstlisting}
cv::copyMakeBorder( //relleno con 0
	image,
	bigger, //puede ser inplace
	0, cv::getOptimalDFTSize(image.rows)-image.rows,
	0, cv::getOptimalDFTSize(image.cols)-image.cols,
	cv::BORDER_CONSTANT
);
			\end{lstlisting}
		%\item hacer compleja %no hace falta
		%	\verb|cv::merge()|
		\item realizar la transformación
			\verb|cv::dft()|
			\begin{itemize}
				\item DFT\_COMPLEX\_OUTPUT si se quiere analizar el espectro,
					sino se utiliza un formato intercalado para almacenar la salida (más eficiente)
				\item DFT\_SCALE divide por el número de píxeles
				\item DFT\_REAL\_OUTPUT si en la transformación directa se usó DFT\_COMPLEX\_OUTPUT %(ver cuándo falla)

			\end{itemize}
		\item aplicar el filtro (que debe estar descentrado, al igual que el espectro)
			\verb|cv::mulSpectrums()|
		\item antitransformar
			\verb|cv::idft()| o
			\verb|cv::dft(DFT_INVERSE)|
		\item eliminar relleno (ROI)
	\end{enumerate}

	Para la visualización se requiere centrar el espectro intercambiando los cuadrantes
	\verb|pdi::centre()|
	%(cátedra provee) (¿no se puede evitar la copia?)

polarToCart(), cartToPolar()
	moverse entre $x+jy$ y $\rho|\theta$.
	no utilizar la misma matriz %(ver bien esto)

%fft
%	dft(DFT_SCALE)
%	idft() o dft(DFT_INVERSA)
%	REAL_OUTPUT si se está completamente seguro (como que K-On es el nombre de la banda) (me dio compleja ¡¿?!)
%	COMPLEX_OUTPUT evitar CSS, si se quiere analizar el espectro
%	ser cuidadoso con los tipos de datos. Trabajar en complejo.
%	Relleno del borde con 0 (habría que eliminarlo después)
%	ver para ocupar fftw3

	\section{Funciones para el cálculo de la TDF}
	\verb|dft() idft()|

	Recibe como parámetros la imagen de entrada y donde se almacenará el resultado, además de banderas que controlan el comportamiento.

	Con dft se puede realizar tanto la transformada directa (flag DFT\_SCALE) o inversa (flag DFT\_INVERSE)
	que se encargará de realizar el escalado correspondiente.

	Para operar con estas funciones se requiere una representación de punto flotante (CV\_32F o CV\_64F)

	Si bien la transformada opera en el plano complejo, se puede forzar que el resultado sea real (si se está seguro de que la entrada es simétrica conjugada) mediante DFT\_REAL\_OUTPUT

	Al realizar los filtros se trabaja en el espacio complejo, forzar con DFT\_COMPLEX\_OUTPUT


	\section{Relleno}
	Con el fin de realizar el cálculo de forma eficiente, se realiza un relleno con 0 a la imagen de entrada.

		\verb|rows = cv::getOptimalDFTSize(image.rows),|\\
		\verb|cols = cv::getOptimalDFTSize(image.cols);|\\
	\verb|cv::copyMakeBorder(image, result, 0, rows-image.rows, 0, cols-image.cols, cv::BORDER_CONSTANT, cv::Scalar::all(0));|

	\section{Definición de filtros en el plano frecuencial}
	Se pueden definir filtros al especificar la magnitud de su espectro.
	Luego, para aplicarlos simplemente se multiplican los espectros del filtro y la imagen.
	Como se está operando en f, se trata de imágenes complejas.

	Para crear la imagen compleja del filtro se tiene
	
	especificación de la magnitud.
	fase 0

	\verb|cv::polarToCart(magnitud, phase, x[0], x[1]);|\\
	\verb|cv::merge(x, 2, result);|

	Luego, para combinar
	\verb|mulSpectrums()|
	(si es justo y necesario, poner el flag en 0)


	\section{Visualizado}
	La imagen de la transformación se encuentra descentrada y es compleja.
	Para centrarla, simplemente se definen 4 rois correspondientes a los cuadrantes y se intercambian según la diagonal.
	realizar el intercambio usando \verb|.copyTo()|

	Como la imagen es compleja no puede visualizarse con \verb|imshow()|. Lo que se visualiza es el logaritmo de su espectro (sumar un offset para evitar calcular $\log(0)$)

	cv::Mat planes[2];
	cv::split(image, planes);

	cv::Mat result;
	cv::magnitude(planes[0], planes[1], result);
	result += 1;
	cv::log(result, result);



	\chapter{Restauración}
		\section{Generación de números aleatorios}
			OpenCV provee funciones \verb|randn()| y \verb|randu()| para distribución normal y uniforme, que llenan toda la matriz de una vez.

			C++11 provee clases para distribuciones uniforme, normal, gamma, bernoulli, exponencial. Reciben como parámetro un RNG.
			Recorrer las matrices elemento a elemento para llenarlas.

			Al sumar la imagen con el ruido se puede superar el rango, se provee la función \verb|pdi::clamp()| para ajustar a los valores extremos.

		\section{Filtros}
			\verb|cv::norm()| para el cálculo de ECM.

			\verb|cv::medianBlur()| para el filtro de mediana.

			Para los de orden también se puede definir una ROI y utilizar \verb|.reshape()| para transformalo en un vector que se puede aplicar \verb|std::sort()|

		\section{Periódico}
			\verb|cv::circle()| y \verb|cv::rectangle()| para dibujar

			se provee la función \verb|pdi::centre()| para desplazar el filtro (o centrar el espectro)

	\chapter{Segmentación}
		Para definir las máscaras puede usarse \verb|Matx22f()| \verb|Matx33f()| que permiten introducir los valores de los píxeles.

		%Se tienen funciones para Sobel y Laplaciano.

		Umbralizado con \verb|threshold()|


		Para crecimiento de regiones se tiene la función \verb|floodfill()| que también permite generar una máscara

		Existe una versión de Hough que ya determina los puntos de inicio y fin de los segmentos
	\chapter{Morfología}
		Funciones \verb|dilate()| y \verb|erode()|

		Creación de máscaras con \verb|Matx22f()| \verb|Matx33f()|, o \verb|Mat_| y usando \verb|operator<<|
		\begin{lstlisting}
Mat C = (Mat_<double>(3,3) <<
	0, 1, 0,
	1, 1, 1,
	0, 1, 0
)
		\end{lstlisting}


\end{document}

